\documentclass[12pt,a4paper]{article}

\usepackage[margin=1in]{geometry}
\usepackage{graphicx}
\usepackage{tikz}
\usepackage{listings}
\usepackage{matlab-prettifier}
\usepackage{xcolor}
\usepackage{Code}
\usepackage{subfigure}
\usepackage{amsmath}
\usepackage{mathtools}
\usepackage{amsfonts}
\usepackage{amssymb}
\usepackage{mathtools}
\usepackage{physics}
\usepackage{blindtext}
\usepackage{hyperref}
\usepackage{fancyhdr}
\usepackage[backend=bibtex]{biblatex}
\addbibresource{References.bib}
\usepackage[bottom,norule]{footmisc}
\usepackage{xepersian}

\settextfont{Yas}


% Define header and footer
\pagestyle{fancy}
\fancyhf{} % Clear default header and footer
\fancyhead[L]{هندسه اطلاعات و کاربردها}
\fancyhead[R]{نظریه اطلاعات, آمار و یادگیری}
\fancyfoot[C]{\thepage}
\fancyfoot[L]{دانشگاه صنعتی شریف}
\renewcommand{\footrulewidth}{1.2pt}
\renewcommand{\headrulewidth}{1.2pt}
\let\origfootrule\footrule
\renewcommand{\footrule}{\iffootnote{}{\origfootrule}}
\renewcommand\footnoterule{\origfootrule}

\hypersetup{
  colorlinks=true,
  linkcolor=blue,
  urlcolor=blue,
  citecolor=blue,
  filecolor=blue,
  linkbordercolor=white,
  urlbordercolor=white,
  citebordercolor=white,
  filebordercolor=white
}

\renewcommand{\thesubsection}{\thesection.\alph{subsection}}

\begin{document}


\begin{titlepage}
    \begin{tikzpicture}[remember picture,overlay]
        \draw[line width=2.4pt]
            ([shift={(1.2cm,-1.2cm)}]current page.north west)
            rectangle
            ([shift={(-1.2cm,1.2cm)}]current page.south east);
    \end{tikzpicture}
    
    \centering
    \vspace*{2.1cm}
    
    {\textbf{\LARGE هندسه اطلاعات و کاربردها}\par}
    
    \vspace{0.7cm}
    
    {\textbf{\Large نظریه اطلاعات, آمار و یادگیری}\par}

    \vspace{0.7cm}

    {\textbf{تاریخ:} \today \par}
    
    \vspace{2.2cm}
    
    \includegraphics[width=0.33\textwidth]{Pictures/university logo.png}
    
    \vspace{1.5cm}
    
    {\textbf{استاد:} دکتر یاسایی\par}
    
    \vspace{0.5cm}
    
    {\textbf{دانشگاه:} صنعتی شریف\par}
    
    \vspace{0.5cm}
    
    {\textbf{دانشکده:} مهندسی برق\par}
    
    \vspace{2cm}
    
    {\textbf{اعضای گروه:}}

    \vspace{0.5cm}
    
    {سپهر حیدری ادواری \hspace{0.2cm} \lr{400109854}}
    \vspace{0.3cm}
    
    {برنا خدابنده\hspace{1.3cm} \lr{400109898}}

\end{titlepage}

\tableofcontents
\pagebreak
\foreach \x in {1,2,4,5,6}{

		\input{Questions/Q\x.tex}
		\pagebreak
	}


\section{تقدیر، نتیجه گیری و منابع}

در تهیه این گزارش، از کار های \lr{Frank nielsen} و به خصوص مقاله \cite{Nielsen_2020} استفاده بسیار شده و در این قسمت از ایشان تقدیر میکنیم، از منابع منبع باز ایشان نیز در بعضی از نتیجه گیری ها و نمودار ها استفاده کرده ایم.

در نهایت، حوزه هندسه اطلاعات، بسیار حوزه بزرگی است و چکیده کردن آن بیشتر از این گزارش، از تمامیت آن میکاهید، این حوزه بسیار حوزه جذابی است و دید های بسیار متفاوتی به علوم اطلاعات، نسبت به روش های کلاسیک میدهد.

ممنون از همراهی شما.

\begin{latin}
	\printbibliography
\end{latin}
\end{document}